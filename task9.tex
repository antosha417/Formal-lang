\documentclass[12pt]{article}
\usepackage[T2A]{fontenc}
\usepackage[utf8]{inputenc}        % Кодировка входного документа;
                                    % при необходимости, вместо cp1251
                                    % можно указать cp866 (Alt-кодировка
                                    % DOS) или koi8-r.

\usepackage[english,russian]{babel} % Включение русификации, русских и
                                    % английских стилей и переносов
%%\usepackage{a4}
%%\usepackage{moreverb}
\usepackage{amsmath,amsfonts,amsthm,amssymb,amsbsy,amstext,amscd,amsxtra,multicol}
\usepackage{indentfirst}
\usepackage{verbatim}
\usepackage{tikz} %Рисование автоматов
\usetikzlibrary{automata,positioning}
\usepackage{multicol} %Несколько колонок
\usepackage{graphicx}
\usepackage[colorlinks,urlcolor=blue]{hyperref}
\usepackage[stable]{footmisc}

%% \voffset-5mm
%% \def\baselinestretch{1.44}
\renewcommand{\theequation}{\arabic{equation}}
\def\hm#1{#1\nobreak\discretionary{}{\hbox{$#1$}}{}}
\newtheorem{Lemma}{Лемма}
\newtheorem{Remark}{Замечание}
%%\newtheorem{Def}{Определение}
\newtheorem{Claim}{Утверждение}
\newtheorem{Cor}{Следствие}
\newtheorem{Theorem}{Теорема}
\theoremstyle{definition}
\newtheorem{Example}{Пример}
\newtheorem*{known}{Теорема}
\def\proofname{Доказательство}
\theoremstyle{definition}
\newtheorem{Def}{Определение}

%% \newenvironment{Example} % имя окружения
%% {\par\noindent{\bf Пример.}} % команды для \begin
%% {\hfill$\scriptstyle\qed$} % команды для \end






%\date{22 июня 2011 г.}
\let\leq\leqslant
\let\geq\geqslant
\def\MT{\mathrm{MT}}
%Обозначения ``ажуром''
\def\BB{\mathbb B}
\def\CC{\mathbb C}
\def\RR{\mathbb R}
\def\SS{\mathbb S}
\def\ZZ{\mathbb Z}
\def\NN{\mathbb N}
\def\FF{\mathbb F}
%date{05 октября 2016 г.}
\let\ra\rightarrow
%греческие буквы
\let\epsilon\varepsilon
\let\es\emptyset
\let\eps\varepsilon
\let\al\alpha
\let\sg\sigma
\let\ga\gamma
\let\ph\varphi
\let\o\omega
\let\ld\lambda
\let\Ld\Lambda
\let\vk\varkappa
\let\Om\Omega
\def\abstractname{}

\def\R{{\cal R}}
\def\A{{\cal A}}
\def\B{{\cal B}}
\def\C{{\cal C}}
\def\D{{\cal D}}
\def\F{{\cal F}}
\let\w\omega

%классы сложности
\def\REG{{\mathsf{REG}}}
\def\CFL{{\mathsf{CFL}}}
\newcounter{problem}
\newcounter{uproblem}
\newcounter{subproblem}
\def\pr{\medskip\noindent\stepcounter{problem}{\bf \theproblem .  }\setcounter{subproblem}{0} }
\def\prp{\medskip\noindent\stepcounter{problem}{\bf Задача \theproblem .  }\setcounter{subproblem}{0} }
\def\prstar{\medskip\noindent\stepcounter{problem}{\bf Задача $\theproblem^*$ .  }\setcounter{subproblem}{0} }
\def\prdag{\medskip\noindent\stepcounter{problem}{\bf Задача $\theproblem^\dagger$ .  }\setcounter{subproblem}{0} }
\def\upr{\medskip\noindent\stepcounter{uproblem}{\bf Упражнение \theuproblem .  }\setcounter{subproblem}{0} }
%\def\prp{\vspace{5pt}\stepcounter{problem}{\bf Задача \theproblem .  } }
%\def\prs{\vspace{5pt}\stepcounter{problem}{\bf \theproblem .*   }
\def\prsub{\medskip\noindent\stepcounter{subproblem}{\rm \thesubproblem .  } }
%прочее
\def\quotient{\backslash\negthickspace\sim}
\begin{document}
\begin{center} {\LARGE Ковальков Антон 577гр} \end{center}
\section*{Задача 1.} 

\begin{tikzpicture}[shorten >=1pt,node distance=2cm,on grid,auto,every node/.style={text centered},initial text=]

	\node [state, initial] (q_0) {$q_0$};
	\node [state, accepting] (q_1) [right = 3cm of q_0] {$q_1$};
	\node [state] (q_2) [right = 3.5cm of q_1] {$q_2$};
	\node [state, accepting] (q_3) [right = 3.5cm of q_2] {$q_3$};
	\path[->]
		
		(q_0) edge [in=30,out=150,loop] node[align=center] {$a, \epsilon/a$\\$b, \epsilon /b$} (q_0)
			  edge node {$c, \epsilon/\epsilon$} (q_1)
			  
		(q_1) edge [in=30,out=150,loop] node[align=center] {$ a,a/\epsilon$\\$b, b/\eps$} (q_1)
			  edge node {$\eps, Z_0/Z_0$} (q_2)
			  edge [in=-140,out=-40] node[below, align=center] {$ a,b/\epsilon$\\$b, a/\eps$} (q_3)

		(q_2) edge node[align=center] {$a, \eps/\eps$\\$b, \epsilon/\eps$} (q_3)

		(q_3) edge [in=30,out=150,loop] node[align=center] {$a, \eps/\eps$\\$b, \epsilon/\eps$} (q_3)

			  
;\end{tikzpicture}

	Если $x \neq y^R$, то рассмотрим варианты слова $w\in L$:
	
	1. $|x|<|y|$ и $x$ является суффиксом $y^R$. Тогда, обработав префикс $w$ длины $2|x|+1$, автомат перейдет в состояние $q_1$, далее по пустому слову и $Z_0$ на стеке перейдет в состояние $q_2$ и обработает все оставшиеся символы из входа.
	
	2. $|x|>|y|$ и $y^R$ является префиксом $x$. Тогда после обработки всего слова, автомат окажется в состоянии $q_1$ с непустым стеком, в котором будет находиться суффикс $x$.
	
	3. $x$ и $y$  имеют произвольную длину и на общем отрезке длины есть позиция $i\ : x[i] \neq y [ |x| - i - 1 ] $. Тогда, обработав часть слова до $|x| - i - 1$  позиции, автомат окажется в состоянии $q_1$, прочитает символ со входа и противоположный символ со стека и перейдет в завершающее состояние $q_3$, в котором обработает оставшуюся часть слова.

	Если $x = y^R$, то обработав все слово, автомат окажется в состоянии $q_1$, по пустому слову и символу $Z$ на стеке перейдет в непринимающее состояние $q_2$ и на этом закончит работу.
%\section*{Задача 2.}
%Пусть $\A$ -- МП автомат принимающий по допускающему состоянию, распознающий язык $L$. Найдём все %состояния $Q$ из которых достижимы допускающие состояния. Сделаем их допускающими. Если на слове $\o$ 
%автомат закончил работу в состоянии $Q$, то существует 
\section*{Задача 3.}
1) Удалим из языка бесплодные символы:\\
\hspace*{1,5cm}$V_0 = \{a, b, \eps\}$ \\
\hspace*{0,5cm}Так как $A \ra \eps, B \ra \eps, C \ra \eps\ \eps \in V_0$, то:\\
\hspace*{1,5cm}$V_1 = \{a, b, \eps\, A, B, C\}$\\
\hspace*{0,5cm}Так как $S \ra A, F \ra aBaaCbA, E \ra A, \{a,b,A,B,C\} \subset V_1$, то:\\
\hspace*{1,5cm}$V_2 = \{a, b, \eps\, A, B, C, S, F, E\}$\\
\hspace*{1,5cm}$V_3 = \{a, b, \eps\, A, B, C, S, F, E\}$\\
\hspace*{0,5cm}Получаем грамматику:
\vspace{-3ex}
\begin{multicols}{2}
	\begin{align*}
		S &\to A \mid B \mid C \mid E \\
		A &\to C  \mid aABC \mid \eps\\
		B &\to bABa \eps\\
	\end{align*}

	\begin{align*}
		C &\to BaAbC \mid \eps\\
		F &\to aBaaCbA\\
		E &\to A\\			
	\end{align*}	
\end{multicols}
\vspace{-2ex}
2) Удалим из языка недостижимые символы:\\
\hspace*{1,5cm}$V_0 = \{S\}$\\
\hspace*{1,5cm}$V_1 = \{S, A, B, C, E\}$\\
\hspace*{1,5cm}$V_2 = \{S, A, B, C, E\}$\\

Получаем грамматику:
\vspace{-3ex}
\begin{multicols}{2}
	\begin{align*}
		S &\to A \mid B \mid C \mid E \\
		A &\to C  \mid aABC \mid \eps\\
		B &\to bABa \eps\\
		C &\to BaAbC \mid \eps\\
		E &\to A\\		
	\end{align*}
\end{multicols}
\vspace{-2ex}

Это и будет приведённая грамматика.

\section*{Задача 4.}
Чтобы вывести слово длины $n$ из НФХ-грамматики, так как все правила неукорачивающие,нужно из стартового символа вывести $n$ нетерминалов, и из каждого нетерминала вывести терминал. На каждом шаге вывода $n$ нетерминалов их количество будет возрастать на единицу. То есть $n$ нетерминалов будет выведено за $n-1$ операцию. Далее необходимо заменить каждый из этих нетерминалов на терминальный символ. Это займёт $n$ операций. Получаем $2n-1$ операция необходима для вывода слова.
\section*{Задача 5.}
1) Удалим из грамматики $\epsilon$-продукции.\\
Найдём $\epsilon$-порождающие символы:\\
$V_0 = \{A\}$\\
$V_1 = \{A, B, C\}$\\
$V_2 = \{A,B,C,S\}$\\
$V_3 = \{A,B,C,S\}$\\
Рассмотрим правило $S \ra ABC$. Так как $\{A,B,C\} \subset V_3$, то получаем правила:\\
\hspace*{0,5cm}$S \ra A \mid B \mid C \mid AB \mid AC \mid BC \mid ABC$.\\
Из правила $S \ra SABC$ получим:\\
\hspace*{0,5cm}$S \ra A \mid B \mid C \mid SA \mid SB \mid SC \mid AB \mid AC \mid BC \mid SAB \mid SAC \mid SBC \mid ABC \mid SABC$.\\
Из правила $A \ra aCB$ получим:\\
\hspace*{0,5cm}$A \ra a \mid aC \mid aB \mid aCB$.\\
Из правила $C \ra aBbA$ получим:\\
\hspace*{0,5cm}$C \ra ab \mid aBb \mid abA \mid aBbA$.\\
\hspace*{0,5cm} Таким образом получим грамматику $G_1$ без $\epsilon$-продукций:\\
$S \ra A \mid B \mid C \mid SA \mid SB \mid SC \mid AB \mid AC \mid BC \mid SAB \mid SAC \mid SBC \mid ABC \mid SABC$\\
$A \ra S \mid B \mid a \mid aC \mid aB \mid aCB$\\
$B \ra ab \mid b \mid A$\\
$C \ra A \mid B \mid ab \mid aBb \mid abA \mid aBbA$\\

Так как исходная грамматика порождала пустое слово, то добавим новый начальный символ $S_1$ и два правила: $S_1 \ra S\ |\ \epsilon$. Получим грамматику: \\
$S_1 \ra S\ |\ \epsilon$\\
$S \ra A \mid B \mid C \mid SA \mid SB \mid SC \mid AB \mid AC \mid BC \mid SAB \mid SAC \mid SBC \mid ABC \mid SABC$\\
$A \ra S \mid B \mid a \mid aC \mid aB \mid aCB$\\
$B \ra ab \mid b \mid A$\\
$C \ra A \mid B \mid ab \mid aBb \mid abA \mid aBbA$\\

2) Удалим из грамматики цепные продукции.\\
Найдём цепные пары:\\
$V_0 = \{(S,S), (A,A), (B,B), (C,C), (S_1, S_1)\}$\\
$V_1 = V_0 \cup \{(S,A), (S,B), (S,C), (A,S), (A,B), (B,A), (C,A), (C,B), (S_1, S)\}$\\
$V_2 = V_1 \cup \{(A,C), (B,S), (C,S), (S_1,A), (S_1,B), (S_1,C)\}$\\
$V_3 = V_2 \cup \{(B,C)\}$\\
$V_4 = V_3$\\

Построим грамматику без цепных продукций:\\
\hspace*{0,5cm}$S_1 \ra SA \mid SB \mid SC \mid AB \mid AC \mid BC \mid SAB \mid SAC \mid SBC \mid ABC \mid SABC \mid a \mid aC \mid aB \mid aCB \mid b \mid ab \mid aBb \mid abA \mid aBbA \mid \epsilon$\\
\hspace*{0,5cm}$S \ra SA \mid SB \mid SC \mid AB \mid AC \mid BC \mid SAB \mid SAC \mid SBC \mid ABC \mid SABC \mid a \mid aC \mid aB \mid aCB \mid ab \mid b \mid aBb \mid abA \mid aBbA$\\
\hspace*{0,5cm}$A \ra SA \mid SB \mid SC \mid AB \mid AC \mid BC \mid SAB \mid SAC \mid SBC \mid ABC \mid SABC \mid a \mid aC \mid aB \mid aCB \mid ab \mid b \mid aBb \mid abA \mid aBbA$\\
\hspace*{0,5cm}$B \ra SA \mid SB \mid SC \mid AB \mid AC \mid BC \mid SAB \mid SAC \mid SBC \mid ABC \mid SABC \mid a \mid aC \mid aB \mid aCB \mid ab \mid b \mid aBb \mid abA \mid aBbA$\\
\hspace*{0,5cm}$C \ra SA \mid SB \mid SC \mid AB \mid AC \mid BC \mid SAB \mid SAC \mid SBC \mid ABC \mid SABC \mid a \mid aC \mid aB \mid aCB \mid ab \mid b \mid aBb \mid abA \mid aBbA$\\

Так как правый части правил у нетерминалов $S, A,B,C$ совпадают, то заменим их нетерминалом $Q$. Получим грамматику:\\
\hspace*{0,5cm}$S_1 \ra QQ \mid QQQ \mid QQQQ \mid a \mid aQ \mid aQQ \mid b \mid ab \mid aQb \mid abQ \mid aQbQ \mid \epsilon$\\
\hspace*{0,5cm}$Q \ra QQ \mid QQQ \mid QQQQ \mid a \mid aQ \mid aQQ \mid ab \mid b \mid aQb \mid abQ \mid aQbQ$\\\\
Бесплодных символов нет, так как $S_1 \ra a$ и $Q \ra a$.\\
Недостижимых символов нет, так как $S_1 \ra QQ$.\\

Построим по получившейся грамматике НФХ:\\
\hspace*{0,5cm}$S_1 \ra QQ \mid Q_2Q \mid Q_2Q_2 \mid a \mid AQ \mid AQ_2 \mid b \mid AB \mid A_QB \mid AB_Q \mid A_QB_Q \mid \epsilon$\\
\hspace*{0,5cm}$Q \ra QQ \mid Q_2Q \mid Q_2Q_2 \mid a \mid AQ \mid AQ_2 \mid AB \mid b \mid A_QB \mid AB_Q \mid A_QB_Q$\\
\hspace*{0,5cm}$A \ra a$\\
\hspace*{0,5cm}$B \ra b$\\
\hspace*{0,5cm}$Q_2 \ra QQ$\\
\hspace*{0,5cm}$A_Q \ra AQ$\\
\hspace*{0,5cm}$B_Q \ra BQ$\\
$\subset$

\end{document}
