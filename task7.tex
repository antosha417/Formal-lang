\documentclass[12pt]{article}
\usepackage[T2A]{fontenc}
\usepackage[utf8]{inputenc}        % Кодировка входного документа;
                                    % при необходимости, вместо cp1251
                                    % можно указать cp866 (Alt-кодировка
                                    % DOS) или koi8-r.

\usepackage[english,russian]{babel} % Включение русификации, русских и
                                    % английских стилей и переносов
%%\usepackage{a4}
%%\usepackage{moreverb}
\usepackage{amsmath,amsfonts,amsthm,amssymb,amsbsy,amstext,amscd,amsxtra,multicol}
\usepackage{indentfirst}
\usepackage{verbatim}
\usepackage{tikz} %Рисование автоматов
\usetikzlibrary{automata,positioning}
\usepackage{multicol} %Несколько колонок
\usepackage{graphicx}
\usepackage[colorlinks,urlcolor=blue]{hyperref}
\usepackage[stable]{footmisc}

%% \voffset-5mm
%% \def\baselinestretch{1.44}
\renewcommand{\theequation}{\arabic{equation}}
\def\hm#1{#1\nobreak\discretionary{}{\hbox{$#1$}}{}}
\newtheorem{Lemma}{Лемма}
\newtheorem{Remark}{Замечание}
%%\newtheorem{Def}{Определение}
\newtheorem{Claim}{Утверждение}
\newtheorem{Cor}{Следствие}
\newtheorem{Theorem}{Теорема}
\theoremstyle{definition}
\newtheorem{Example}{Пример}
\newtheorem*{known}{Теорема}
\def\proofname{Доказательство}
\theoremstyle{definition}
\newtheorem{Def}{Определение}

%% \newenvironment{Example} % имя окружения
%% {\par\noindent{\bf Пример.}} % команды для \begin
%% {\hfill$\scriptstyle\qed$} % команды для \end






%\date{22 июня 2011 г.}
\let\leq\leqslant
\let\geq\geqslant
\def\MT{\mathrm{MT}}
%Обозначения ``ажуром''
\def\BB{\mathbb B}
\def\CC{\mathbb C}
\def\RR{\mathbb R}
\def\SS{\mathbb S}
\def\ZZ{\mathbb Z}
\def\NN{\mathbb N}
\def\FF{\mathbb F}
%date{05 октября 2016 г.}
\let\ra\rightarrow
%греческие буквы
\let\epsilon\varepsilon
\let\es\emptyset
\let\eps\varepsilon
\let\al\alpha
\let\sg\sigma
\let\ga\gamma
\let\ph\varphi
\let\o\omega
\let\ld\lambda
\let\Ld\Lambda
\let\vk\varkappa
\let\Om\Omega
\def\abstractname{}

\def\R{{\cal R}}
\def\A{{\cal A}}
\def\B{{\cal B}}
\def\C{{\cal C}}
\def\D{{\cal D}}
\def\F{{\cal F}}
\let\w\omega

%классы сложности
\def\REG{{\mathsf{REG}}}
\def\CFL{{\mathsf{CFL}}}
\newcounter{problem}
\newcounter{uproblem}
\newcounter{subproblem}
\def\pr{\medskip\noindent\stepcounter{problem}{\bf \theproblem .  }\setcounter{subproblem}{0} }
\def\prp{\medskip\noindent\stepcounter{problem}{\bf Задача \theproblem .  }\setcounter{subproblem}{0} }
\def\prstar{\medskip\noindent\stepcounter{problem}{\bf Задача $\theproblem^*$ .  }\setcounter{subproblem}{0} }
\def\prdag{\medskip\noindent\stepcounter{problem}{\bf Задача $\theproblem^\dagger$ .  }\setcounter{subproblem}{0} }
\def\upr{\medskip\noindent\stepcounter{uproblem}{\bf Упражнение \theuproblem .  }\setcounter{subproblem}{0} }
%\def\prp{\vspace{5pt}\stepcounter{problem}{\bf Задача \theproblem .  } }
%\def\prs{\vspace{5pt}\stepcounter{problem}{\bf \theproblem .*   }
\def\prsub{\medskip\noindent\stepcounter{subproblem}{\rm \thesubproblem .  } }
%прочее
\def\quotient{\backslash\negthickspace\sim}
\begin{document}
\begin{center} {\LARGE Ковальков Антон 577гр} \end{center}
\section*{Задача  1.}
1) Построим автомат по такому принципу:\\
\hspace*{1cm} a) начальное состояние принимающее так как $\epsilon$ лежит в языке.\\
\hspace*{1cm} b) если на стеке только $Z_0$ кладём букву из слова на стек.\\
\hspace*{1cm} c) если следующая буква $\sigma$ противоположна той что лежит на стеке($\lambda$), то убрием $\lambda$ со стека. Если же $\sigma = \lambda$, то кладём на стек $\sigma\lambda$. \\
\hspace*{1cm} d) Если букв $a$ и букв $b$ в слове одинаковое количество, то повтряя пункты b и c мы обработаем всё слово и на стеке останется только $Z_0$. Уберём $Z_0$ со стека и перейдём в конечное состояние.\\
\begin{tikzpicture}[shorten >=1pt,node distance=2cm,on grid,auto,every node/.style={text centered},initial text=]
	\node [state, initial, accepting] (q_0) {$q_0$};
	\node [state] (q_1) [right = 4cm of q_0] {$q_1$};
	\node [state, accepting] (q_2) [right = 4cm of q_1] {$q_2$};
	\path[->]
		
		(q_1) edge [in=-30,out=-150,loop] node [below] {$b,b/bb\ b,a/\eps\ b,Z_0/bZ_0$} (q_0)
			  edge [in=30,out=150,loop] node {$a,a/aa\ a, b/\eps\ a,Z_0/aZ_0$} (q_0)
			  edge node {$\eps, Z_0/\epsilon$} (q_2)
		(q_0) edge node {$a,Z_0/aZ_0$} (q_1)
			  edge node [below] {$b,Z_0/bZ_0$} (q_1)
			  
;\end{tikzpicture}\\


\section*{Задача  2.}
1) Построим автомат по такому принципу:\\
\hspace*{1cm} a) начальное состояние принимающее так как $\epsilon$ лежит в языке.\\
\hspace*{1cm} b) если на стеке только $Z_0$ кладём букву из слова на стек.\\
\hspace*{1cm} c) если следующая буква $\sigma$ закрывающая скобка а на стеке лежит открывающая такого же типа $\lambda$, то убрием $\lambda$ со стека. Если же $\sigma$ открывающая скобка, то кладём её на стек. \\
\hspace*{1cm} d) Если если слово это правильна скобочная последовательность, то повтряя пункты b и c мы обработаем всё слово и на стеке останется только $Z_0$. Уберём $Z_0$ со стека и перейдём в конечное состояние.\\

\begin{tikzpicture}[shorten >=1pt,node distance=2cm,on grid,auto,every node/.style={text centered},initial text=]
	\node [state, initial, accepting] (q_0) {$q_0$};
	\node [state] (q_1) [right = 4cm of q_0] {$q_1$};
	\node [state, accepting] (q_2) [right = 4cm of q_1] {$q_2$};
	\path[->]
		(q_0) edge node {$(,Z_0/(Z_0$} (q_1)
			  edge node [below] {$[,Z_0/[Z_0$} (q_1)
		(q_1) edge [in=-30,out=-150,loop] node [below] {$(,\eps/(\ \ ),(/\eps$} (q_1)
			  edge [in=30,out=150,loop] node {$[,\eps/[\ \ ], [/\eps$} (q_1)
			  edge node [below] {$\eps,Z_0/\eps$} (q_2)
;\end{tikzpicture}\\


\section*{Задача  3.}

1. Состояние $q_0$ принимающее, так как $\epsilon\in L$;

2. Если длина слова 1, то переходим в принимающее состояние $q_3$;

3. Если длина слова больше 1, то по первой букве перейдем в состояние $q_1$, и пока не достигнем позиции $[\dfrac{|\omega|}{2}] + 1$, будем класть буквы слова в стек. 

4. Если длина слова делится на 2, то из состояния $q_1$ переходим в $q_2$, убирая со стека букву, если она равна обрабатываемой. Если длина слова нечетна, то переходим, не изменяя стек.

5. Если очередная буква в обрабатываемом слове равна той, что лежит на вершине стека, то убираем букву со стека.

6. Если после проделанных операций в стеке остался только символ $Z_0$, то переходим в принимающее состояние. Если в стеке осталось что-то еще, то слово было не палиндромом и автомат его не примет.
\begin{tikzpicture}[shorten >=1pt,node distance=2cm,on grid,auto,every node/.style={text centered},initial text=]
	\node [state, initial, accepting] (q_0) {$q_0$};
	\node [state] (q_1) [right = 4cm of q_0] {$q_1$};
	\node [state] (q_2) [right = 4cm of q_1] {$q_2$};
	\node [state, accepting] (q_3) [right = 4cm of q_2] {$q_3$};
	\path[->]
		(q_0) edge node {$a,Z_0/aZ_0$} (q_1)
			  edge node [below] {$b,Z_0/bZ_0$} (q_1)
			  edge [in=130,out=50] node [below] {$a, Z_0/\eps\ \ b,Z_0/\epsilon$} (q_3)
		(q_1) edge [in=30,out=150,loop] node {$a, \eps / a\ \ b, \eps / b$} (q_1)
			  edge node [below] {$a,a/\eps\ \ b, b/\eps$} (q_2)
			  edge node {$a,\eps/\eps\ \ b, \eps/\eps$} (q_2)
		(q_2) edge node {$\eps, Z_0/\epsilon$} (q_3)
			  edge [in=30,out=150,loop] node {$a,a/\eps\ \ b, b/\eps$} (q_2)
;\end{tikzpicture}\\


\section*{Задача  4.}
1. $L = \{a^nb^nc^k, \forall n, k \geq 0 \} \cup \{a^nb^kc^n, \forall n, k \geq 0 \}$

	$S \ra AB\ |\ C$
	
	$A\ra aAb\ |\ \eps$
	
	$B\ra cB\ |\ \epsilon$ 
	
	$C\ra aCc\ |\ D$
	
	$D\ra bD\ |\ \epsilon$
	
	Из правила $S \ra AB$ получается любое слово вида $a^nb^nc^k$. Для этого нужно $n$ раз применить правило $A\ra aAb$, потом правило $A\ra \epsilon$.  Затем $k$ раз применить правило $B\ra cB$, потом правило $B\ra \epsilon$.
		
	Из правила $S \ra C$ получается любое слово вида $a^nb^kc^n$. Для этого нужно $n$ раз применить правило $C\ra aCc$, потом правило $C\ra D$.  Затем $k$ раз применить правило $D\ra bD$, потом правило $D\ra \epsilon$.

\section*{Задача  5.}
Воспользуемся алгоритмом из книги "Введение в теорию автоматов, языков и вычислений".
В задаче 1 был построен автомат допускающий по принимающему состоянию.\\
\begin{tikzpicture}[shorten >=1pt,node distance=2cm,on grid,auto,every node/.style={text centered},initial text=]
	\node [state, initial, accepting] (q_0) {$q_0$};
	\node [state] (q_1) [right = 4cm of q_0] {$q_1$};
	\node [state, accepting] (q_2) [right = 4cm of q_1] {$q_2$};
	\path[->]
		
		(q_1) edge [in=-30,out=-150,loop] node [below] {$b,b/bb\ b,a/\eps\ b,Z_0/bZ_0$} (q_0)
			  edge [in=30,out=150,loop] node {$a,a/aa\ a, b/\eps\ a,Z_0/aZ_0$} (q_0)
			  edge node {$\eps, Z_0/\epsilon$} (q_2)
		(q_0) edge node {$a,Z_0/aZ_0$} (q_1)
			  edge node [below] {$b,Z_0/bZ_0$} (q_1)
			  
;\end{tikzpicture}\\
Добавим в него состояние $q_3$ в которое будет переход из каждого принимающего состояния по $\eps$,  и будем опусташать стек в состоянии $q_3$.
\begin{tikzpicture}[shorten >=1pt,node distance=2cm,on grid,auto,every node/.style={text centered},initial text=]
	\node [state, initial] (q_0) {$q_0$};
	\node [state] (q_1) [right = 3.5cm of q_0] {$q_1$};
	\node [state] (q_2) [right = 3.5cm of q_1] {$q_2$};
	\node [state] (q_3) [right = 3.5cm of q_2] {$q_3$};
	\path[->]
		
		(q_1) edge [in=-30,out=-150,loop] node [below] {$b,b/bb\ b,a/\eps\ b,Z_0/bZ_0$} (q_0)
			  edge [in=30,out=150,loop] node {$a,a/aa\ a, b/\eps\ a,Z_0/aZ_0$} (q_0)
			  edge node {$\eps, Z_0/\epsilon$} (q_2)
		(q_0) edge node {$a,Z_0/aZ_0$} (q_1)
			  edge node [below] {$b,Z_0/bZ_0$} (q_1)
			  edge [in=100,out=85] node {$\eps,$ любой$/\eps$} (q_3)
		(q_2) edge node {$\eps,$ любой$/\eps$} (q_3)
		(q_3) edge [in=-60,out=-150,loop] node [below] {$\eps,$ любой$/\eps$} (q_3)
			  
;\end{tikzpicture}\\

\end{document}
