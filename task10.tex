\documentclass[12pt]{article}
\usepackage[T2A]{fontenc}
\usepackage[utf8]{inputenc}        % Кодировка входного документа;
                                    % при необходимости, вместо cp1251
                                    % можно указать cp866 (Alt-кодировка
                                    % DOS) или koi8-r.

\usepackage[english,russian]{babel} % Включение русификации, русских и
                                    % английских стилей и переносов
%%\usepackage{a4}
%%\usepackage{moreverb}
\usepackage{amsmath,amsfonts,amsthm,amssymb,amsbsy,amstext,amscd,amsxtra,multicol}
\usepackage{indentfirst}
\usepackage{verbatim}
\usepackage{tikz} %Рисование автоматов
\usetikzlibrary{automata,positioning}
\usepackage{multicol} %Несколько колонок
\usepackage{graphicx}
\usepackage[colorlinks,urlcolor=blue]{hyperref}
\usepackage[stable]{footmisc}

%% \voffset-5mm
%% \def\baselinestretch{1.44}
\renewcommand{\theequation}{\arabic{equation}}
\def\hm#1{#1\nobreak\discretionary{}{\hbox{$#1$}}{}}
\newtheorem{Lemma}{Лемма}
\newtheorem{Remark}{Замечание}
%%\newtheorem{Def}{Определение}
\newtheorem{Claim}{Утверждение}
\newtheorem{Cor}{Следствие}
\newtheorem{Theorem}{Теорема}
\theoremstyle{definition}
\newtheorem{Example}{Пример}
\newtheorem*{known}{Теорема}
\def\proofname{Доказательство}
\theoremstyle{definition}
\newtheorem{Def}{Определение}

%% \newenvironment{Example} % имя окружения
%% {\par\noindent{\bf Пример.}} % команды для \begin
%% {\hfill$\scriptstyle\qed$} % команды для \end



\let\leq\leqslant
\let\geq\geqslant
\def\MT{\mathrm{MT}}
%Обозначения ``ажуром''
\def\BB{\mathbb B}
\def\CC{\mathbb C}
\def\RR{\mathbb R}
\def\SS{\mathbb S}
\def\ZZ{\mathbb Z}
\def\NN{\mathbb N}
\def\FF{\mathbb F}
%date{05 октября 2016 г.}
\let\ra\rightarrow
%греческие буквы
\let\epsilon\varepsilon
\let\es\emptyset
\let\eps\varepsilon
\let\al\alpha
\let\sg\sigma
\let\ga\gamma
\let\ph\varphi
\let\o\omega
\let\ld\lambda
\let\Ld\Lambda
\let\vk\varkappa
\let\Om\Omega
\def\abstractname{}

\def\R{{\cal R}}
\def\A{{\cal A}}
\def\B{{\cal B}}
\def\C{{\cal C}}
\def\D{{\cal D}}
\def\F{{\cal F}}
\let\w\omega

%классы сложности
\def\REG{{\mathsf{REG}}}
\def\CFL{{\mathsf{CFL}}}
\newcounter{problem}
\newcounter{uproblem}
\newcounter{subproblem}
\def\pr{\medskip\noindent\stepcounter{problem}{\bf \theproblem .  }\setcounter{subproblem}{0} }
\def\prp{\medskip\noindent\stepcounter{problem}{\bf Задача \theproblem .  }\setcounter{subproblem}{0} }
\def\prstar{\medskip\noindent\stepcounter{problem}{\bf Задача $\theproblem^*$ .  }\setcounter{subproblem}{0} }
\def\prdag{\medskip\noindent\stepcounter{problem}{\bf Задача $\theproblem^\dagger$ .  }\setcounter{subproblem}{0} }
\def\upr{\medskip\noindent\stepcounter{uproblem}{\bf Упражнение \theuproblem .  }\setcounter{subproblem}{0} }
%\def\prp{\vspace{5pt}\stepcounter{problem}{\bf Задача \theproblem .  } }
%\def\prs{\vspace{5pt}\stepcounter{problem}{\bf \theproblem .*   }
\def\prsub{\medskip\noindent\stepcounter{subproblem}{\rm \thesubproblem .  } }
%прочее
\def\quotient{\backslash\negthickspace\sim}
\begin{document}
\begin{center} {\LARGE Ковальков Антон 577гр} \end{center}
\section*{Задача 1.}
Левый и правый разбор в данном случае совпадают, так как грамматика однозначна и в словах из цепочки вывода встречается не более одного нетерминала.

\begin{tikzpicture}
	\node {$E$} %root
		child{ node   {$T$}
		 	child{ node {$F$}
				child{ node{$($} }
				child{ node{$E$} 
					child{ node{$T$}
						child{ node{$F$}
							child{ node{$($}}
							child{ node{$E$}
								child{ node{$T$}
									child{ node{$F$}
										child{ node{$a$} }									
									}								
								}			
							}
							child{ node{$)$}}
						}
					}
				}
				child{ node{$)$} }			
			}
	};
\end{tikzpicture}


\section*{Задача 2.}
\noindent Вычислим функцию First для каждого нетерминала.\\
\begin{tabular}{|c||c|}
\hline
&$S$ \\
\hline
$F_0$ & $\{\epsilon\}$  \\
$F_1$ & $\{\epsilon, 0, 1\}$ \\
$F_2$ & $\{\epsilon, 0, 1\}$ \\
\hline
\end{tabular}\\

\noindent Вычислим функцию Follow для каждого нетерминала.\\
\begin{tabular}{|c||c|}
\hline
&$S$ \\
\hline
$F_0$ & $\varnothing$  \\
$F_1$ & $\varnothing$ \\
\hline
\end{tabular}\\

\noindent Построим таблицу для детерминированного левого анализатора.\\
\begin{tabular}{|c||c|c|c|}
\hline
&$0$ &$1$ &$\$$\\
\hline
$S$ & $0S$ & $1S$ &$\epsilon$ \\
\hline
\end{tabular}\\

\noindent По таблице построим детерминированный левый анализатор.\\

\begin{tikzpicture}[shorten >=1pt,node distance=2cm,on grid,auto,every node/.style={text centered},initial text=]

	\node [state, initial] (q_0) {$q_0$};
	\path[->]
		
		(q_0) edge [in=30,out=150,loop] node[align=center] {$0,0 /\epsilon$\\$1,1 /\epsilon$} (q_0)
			  edge [in=-30,out=-150,loop] node[below,align=center] {$0,S/0S$\\$1,S/1S$\\$\$, S/\epsilon$} (q_0)

			  
;\end{tikzpicture}

\section*{Задача 4.}
\noindent \textbf{Теорема:} \\
Грамматика является $LL(k)$-грамматикой тогда и только тогда, когда для любых двух правил $A \ra \beta, A \ra \gamma$, $FIRST_k(\gamma\alpha)\cap FIRST_k(\beta\alpha) = \varnothing $ для таких $\alpha$, что $ S \Rightarrow^{*}_l wA\alpha$.

Допустим, что заданная грамматика $LL(1)$. Рассмотрим правило $S \ra bAba.$ Тогда так как $A \ra b$ и $A \ra \epsilon$, то по теореме\\ $FIRST(bba) \cap FIRST(\epsilon ba) = \varnothing$, что не так, так как\\ $FIRST(bba) \cap FIRST(\epsilon ba) = \{b\}$.\\
Значит заданная грамматика не является $LL(1)$.

Рассмотрим сентенциальные цепочки выводимые из $S$:\\ $\{aAaa, bAba, abaa, aaa, bbba, bba\}$. Проверим, что для каждой выполнена теорема. Для первой: $FIRST_2(baa) \cap FIRST_2(\epsilon aa) = \varnothing$, для второй $FIRST_2(bba) \cap FIRST_2(\epsilon ba) = \varnothing$. В остальных цепочках нет нетерминалов, то есть они не подходят под условие теоремы.\\
Значит теорема выполняется и заданная грамматика $LL-2$.\\


Вычислим $FIRST_2$ для каждого нетерминала:\\
\begin{tabular}{|c||c|c|}
\hline
&$S$ &$A$\\
\hline
$F_0$ & $\varnothing$  & $\{b, \eps\}$\\
$F_1$ & $\{ab, aa, bb\}$  & $\{b, \eps\}$\\
$F_2$ & $\{ab, aa, bb\}$  & $\{b, \eps\}$\\
\hline
\end{tabular}\\

Вычислим $FOLLOW_2$ для каждого нетерминала:\\
\begin{tabular}{|c||c|c|}
\hline
&$S$ &$A$\\
\hline
$F_0$ & $\varnothing$  & $\varnothing$\\
$F_1$ & $\varnothing$  & $\{aa, ba\}$\\
$F_2$ & $\varnothing$  & $\{aa, ba\}$\\

\hline
\end{tabular}\\

\end{document}
