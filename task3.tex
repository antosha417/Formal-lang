\documentclass[12pt]{article}
\usepackage[T2A]{fontenc}
\usepackage[utf8]{inputenc}        % Кодировка входного документа;
                                    % при необходимости, вместо cp1251
                                    % можно указать cp866 (Alt-кодировка
                                    % DOS) или koi8-r.

\usepackage[english,russian]{babel} % Включение русификации, русских и
                                    % английских стилей и переносов
%%\usepackage{a4}
%%\usepackage{moreverb}
\usepackage{amsmath,amsfonts,amsthm,amssymb,amsbsy,amstext,amscd,amsxtra,multicol}
\usepackage{indentfirst}
\usepackage{verbatim}
\usepackage{tikz} %Рисование автоматов
\usetikzlibrary{automata,positioning}
\usepackage{multicol} %Несколько колонок
\usepackage{graphicx}
\usepackage[colorlinks,urlcolor=blue]{hyperref}
\usepackage[stable]{footmisc}

%% \voffset-5mm
%% \def\baselinestretch{1.44}
\renewcommand{\theequation}{\arabic{equation}}
\def\hm#1{#1\nobreak\discretionary{}{\hbox{$#1$}}{}}
\newtheorem{Lemma}{Лемма}
\theoremstyle{definiton}
\newtheorem{Remark}{Замечание}
%%\newtheorem{Def}{Определение}
\newtheorem{Claim}{Утверждение}
\newtheorem{Cor}{Следствие}
\newtheorem{Theorem}{Теорема}
\theoremstyle{definition}
\newtheorem{Example}{Пример}
\newtheorem*{known}{Теорема}
\def\proofname{Доказательство}
\theoremstyle{definition}
\newtheorem{Def}{Определение}

%% \newenvironment{Example} % имя окружения
%% {\par\noindent{\bf Пример.}} % команды для \begin
%% {\hfill$\scriptstyle\qed$} % команды для \end






%\date{22 июня 2011 г.}
\let\leq\leqslant
\let\geq\geqslant
\def\MT{\mathrm{MT}}
%Обозначения ``ажуром''
\def\BB{\mathbb B}
\def\CC{\mathbb C}
\def\RR{\mathbb R}
\def\SS{\mathbb S}
\def\ZZ{\mathbb Z}
\def\NN{\mathbb N}
\def\FF{\mathbb F}
%греческие буквы
\let\epsilon\varepsilon
\let\es\emptyset
\let\eps\varepsilon
\let\al\alpha
\let\sg\sigma
\let\ga\gamma
\let\ph\varphi
\let\o\omega
\let\ld\lambda
\let\Ld\Lambda
\let\vk\varkappa
\let\Om\Omega
\def\abstractname{}

\def\R{{\cal R}}
\def\A{{\cal A}}
\def\B{{\cal B}}
\def\C{{\cal C}}
\def\D{{\cal D}}
\def\F{{\cal F}}
\let\w\omega

%классы сложности
\def\REG{{\mathsf{REG}}}
\def\CFL{{\mathsf{CFL}}}
\newcounter{problem}
\newcounter{uproblem}
\newcounter{subproblem}
\def\pr{\medskip\noindent\stepcounter{problem}{\bf \theproblem .  }\setcounter{subproblem}{0} }
\def\prp{\medskip\noindent\stepcounter{problem}{\bf Задача \theproblem .  }\setcounter{subproblem}{0} }
\def\prstar{\medskip\noindent\stepcounter{problem}{\bf Задача $\theproblem^*$ .  }\setcounter{subproblem}{0} }
\def\prdag{\medskip\noindent\stepcounter{problem}{\bf Задача $\theproblem^\dagger$ .  }\setcounter{subproblem}{0} }
\def\upr{\medskip\noindent\stepcounter{uproblem}{\bf Упражнение \theuproblem .  }\setcounter{subproblem}{0} }
%\def\prp{\vspace{5pt}\stepcounter{problem}{\bf Задача \theproblem .  } }
%\def\prs{\vspace{5pt}\stepcounter{problem}{\bf \theproblem .*   }
\def\prsub{\medskip\noindent\stepcounter{subproblem}{\rm \thesubproblem .  } }
%прочее
\def\quotient{\backslash\negthickspace\sim}
\begin{document}
\begin{center} {\LARGE Ковальков Антон 577гр} \end{center}
\section*{Задача  1.}

Построим НКА $\A$:

\begin{center}
\begin{tikzpicture}[scale=0.2]
\tikzstyle{every node}+=[inner sep=0pt]
\draw [black] (18.1,-24.2) circle (3);
\draw (18.1,-24.2) node {$q_0$};
\draw [black] (28.1,-24.2) circle (3);
\draw (28.1,-24.2) node {$q_1$};
\draw [black] (38.9,-24.2) circle (3);
\draw (38.9,-24.2) node {$q_2$};
\draw [black] (50,-24.2) circle (3);
\draw (50,-24.2) node {$q_3$};
\draw [black] (50,-24.2) circle (2.4);
\draw [black] (21.1,-24.2) -- (25.1,-24.2);
\fill [black] (25.1,-24.2) -- (24.3,-23.7) -- (24.3,-24.7);
\draw (23.1,-24.7) node [below] {$1$};
\draw [black] (30.806,-22.933) arc (106.19884:73.80116:9.657);
\fill [black] (36.19,-22.93) -- (35.57,-22.23) -- (35.29,-23.19);
\draw (33.5,-22.05) node [above] {$0$};
\draw [black] (36.088,-25.222) arc (-77.32299:-102.67701:11.792);
\fill [black] (36.09,-25.22) -- (35.2,-24.91) -- (35.42,-25.88);
\draw (33.5,-26.01) node [below] {$1$};
\draw [black] (41.646,-23.016) arc (105.26337:74.73663:10.653);
\fill [black] (47.25,-23.02) -- (46.61,-22.32) -- (46.35,-23.29);
\draw (44.45,-22.14) node [above] {$0$};
\draw [black] (47.233,-25.335) arc (-75.45364:-104.54636:11.08);
\fill [black] (47.23,-25.34) -- (46.33,-25.05) -- (46.58,-26.02);
\draw (44.45,-26.19) node [below] {$1$};
\draw [black] (10.4,-24.2) -- (15.1,-24.2);
\fill [black] (15.1,-24.2) -- (14.3,-23.7) -- (14.3,-24.7);
\draw [black] (16.777,-21.52) arc (234:-54:2.25);
\draw (18.1,-16.95) node [above] {$0,\mbox{ }1$};
\fill [black] (19.42,-21.52) -- (20.3,-21.17) -- (19.49,-20.58);
\end{tikzpicture}
\end{center}

Докажем, что $\A$ распознаёт язык $L_3$.\\
1) $\forall \o \in L_3$  $\o = \o_1 1 \o_2$, где $|\o_2|=2$\\
На слове $\o_1$ в множестве состояний автомата будет состояние $q_0$. После $1$ во множестве состояний гарантированно будет $q_1$. И, наконец, так как  $|\o_2|=2$ после обработки подслова $\o_2$ во множестве состояний автомата будет принимающее состояние $q_3$. Значит автомат принимает все слова из $L_3$.\\ 
2) Докажем теперь, что автомат не принимает слова у которых на третьем с конца месте не $1$. Очевидно, что слова длины меньшей 3 автомат не принимает. Так же автомат не принимает слова длины 3 начинающиеся на 0. Разделим все слова длины большей 3  у которых на третьем с конца месте не $1$ на 3 части :
 $\o = \o_10\o_2$, где $|\o_2|=2$. Чтобы автомат завершил работу он должен оказаться в состоянии $q_3$. Что не возможно.\\
 
По автомату $\A$ построим ДКА $\B$:
\begin{tabular}{cccc}
Макросост. & сост НКА & 0 & 1 \\
$Q_0$ & $q_0$ & $Q_0$ & $Q_1$ \\
$Q_1$ & $q_0, q_1$ & $Q_2$ & $Q_3$ \\
$Q_2$ & $q_0, q_2$ & $Q_4$ & $Q_5$ \\
$Q_3$ & $q_0, q_1, q_2$ & $Q_6$ & $Q_7$ \\
$Q_4$ & $q_0, q_3$ & $Q_0$ & $Q_1$ \\
$Q_5$ & $q_0, q_1, q_3$ & $Q_2$ & $Q_3$ \\
$Q_6$ & $q_0, q_2, q_3$ & $Q_4$ & $Q_5$ \\
$Q_7$ & $q_0, q_1, q_2, q_3$ & $Q_6$ & $Q_7$ \\
\end{tabular}

\begin{center}
\begin{tikzpicture}[scale=0.2]
\tikzstyle{every node}+=[inner sep=0pt]
\draw [black] (10,-9.2) circle (3);
\draw (10,-9.2) node {$Q_0$};
\draw [black] (21.6,-9.2) circle (3);
\draw (21.6,-9.2) node {$Q_1$};
\draw [black] (33.2,-9.2) circle (3);
\draw (33.2,-9.2) node {$Q_2$};
\draw [black] (47.8,-10.8) circle (3);
\draw (47.8,-10.8) node {$Q_3$};
\draw [black] (18.4,-22.9) circle (3);
\draw (18.4,-22.9) node {$Q_4$};
\draw [black] (18.4,-22.9) circle (2.4);
\draw [black] (38.5,-20.2) circle (3);
\draw (38.5,-20.2) node {$Q_5$};
\draw [black] (38.5,-20.2) circle (2.4);
\draw [black] (39.7,-32.6) circle (3);
\draw (39.7,-32.6) node {$Q_6$};
\draw [black] (39.7,-32.6) circle (2.4);
\draw [black] (60.4,-26.3) circle (3);
\draw (60.4,-26.3) node {$Q_7$};
\draw [black] (60.4,-26.3) circle (2.4);
\draw [black] (3.8,-9.2) -- (7,-9.2);
\fill [black] (7,-9.2) -- (6.2,-8.7) -- (6.2,-9.7);
\draw [black] (13,-9.2) -- (18.6,-9.2);
\fill [black] (18.6,-9.2) -- (17.8,-8.7) -- (17.8,-9.7);
\draw (15.8,-9.7) node [below] {$1$};
\draw [black] (8.677,-6.52) arc (234:-54:2.25);
\draw (10,-1.95) node [above] {$0$};
\fill [black] (11.32,-6.52) -- (12.2,-6.17) -- (11.39,-5.58);
\draw [black] (24.6,-9.2) -- (30.2,-9.2);
\fill [black] (30.2,-9.2) -- (29.4,-8.7) -- (29.4,-9.7);
\draw (27.4,-9.7) node [below] {$0$};
\draw [black] (23.9,-7.279) arc (124.984:48.02672:17.592);
\fill [black] (45.75,-8.61) -- (45.49,-7.71) -- (44.82,-8.45);
\draw (35.13,-3.58) node [above] {$1$};
\draw [black] (31.554,-11.707) arc (-35.68936:-58.73128:35.912);
\fill [black] (21.03,-21.45) -- (21.97,-21.46) -- (21.45,-20.61);
\draw (27.8,-17.6) node [below] {$0$};
\draw [black] (35.869,-18.793) arc (-128.32062:-180.22816:8.443);
\fill [black] (35.87,-18.79) -- (35.55,-17.9) -- (34.93,-18.69);
\draw (32.79,-16.9) node [left] {$1$};
\draw [black] (47.322,-13.761) arc (-10.81743:-29.94864:52.268);
\fill [black] (41.27,-30.04) -- (42.1,-29.6) -- (41.24,-29.1);
\draw (45.73,-22.97) node [right] {$0$};
\draw [black] (50.349,-12.379) arc (54.94607:23.26947:26.215);
\fill [black] (59.37,-23.48) -- (59.52,-22.55) -- (58.6,-22.94);
\draw (56.19,-15.87) node [right] {$1$};
\draw [black] (16.287,-20.773) arc (-138.19198:-158.77987:28.634);
\fill [black] (10.94,-12.05) -- (10.76,-12.98) -- (11.69,-12.61);
\draw (12.58,-17.93) node [left] {$0$};
\draw [black] (22.205,-12.132) arc (5.34536:-31.63985:13.61);
\fill [black] (22.21,-12.13) -- (21.78,-12.98) -- (22.78,-12.88);
\draw (22.66,-16.9) node [right] {$1$};
\draw [black] (35.309,-11.326) arc (38.82216:12.62907:14.451);
\fill [black] (35.31,-11.33) -- (35.42,-12.26) -- (36.2,-11.64);
\draw (37.78,-13.04) node [right] {$0$};
\draw [black] (40.61,-18.07) -- (45.69,-12.93);
\fill [black] (45.69,-12.93) -- (44.77,-13.15) -- (45.48,-13.85);
\draw (43.67,-16.98) node [right] {$1$};
\draw [black] (36.778,-31.924) arc (-104.69226:-124.27669:51.522);
\fill [black] (20.83,-24.66) -- (21.21,-25.52) -- (21.77,-24.7);
\draw (27.51,-29.48) node [below] {$0$};
\draw [black] (38.835,-29.729) arc (-166.75121:-182.19371:24.475);
\fill [black] (38.2,-23.18) -- (37.67,-23.96) -- (38.67,-24);
\draw (37.66,-26.56) node [left] {$1$};
\draw [black] (57.718,-27.644) arc (-64.93744:-81.20754:55.558);
\fill [black] (42.68,-32.22) -- (43.54,-32.59) -- (43.39,-31.61);
\draw (51.2,-31.01) node [below] {$0$};
\draw [black] (61.591,-23.559) arc (184.23636:-103.76364:2.25);
\draw (66.1,-20.5) node [right] {$1$};
\fill [black] (63.3,-25.58) -- (64.13,-26.02) -- (64.06,-25.02);
\end{tikzpicture}
\end{center}

 
\section*{Задача 3.}
1. Язык $A = \{a^{2013k+5}, k \geq 0\}$ Регулярный, так как можно построить автомат, который его распознаёт. 
\begin{center}
\begin{tikzpicture}[scale=0.2]
\tikzstyle{every node}+=[inner sep=0pt]
\draw [black] (10.1,-12.9) circle (3);
\draw (10.1,-12.9) node {$q_0$};
\draw [black] (32.7,-12.9) circle (3);
\draw (32.7,-12.9) node {$q_2$};
\draw [black] (43.2,-12.9) circle (3);
\draw (43.2,-12.9) node {$q_3$};
\draw [black] (54.5,-12.9) circle (3);
\draw (54.5,-12.9) node {$q_4$};
\draw [black] (66.8,-33.4) circle (3);
\draw (66.8,-33.4) node {$Q_1$};
\draw [black] (56,-33.4) circle (3);
\draw (56,-33.4) node {$Q_2$};
%\draw [black] (44.4,-33.4) circle (3);
\draw (44.4,-33.4) node {$\dots$};
\draw [black] (31.9,-33.4) circle (3);
\draw (31.9,-33.4) node {$Q_{2012}$};
\draw [black] (21,-12.9) circle (3);
\draw (21,-12.9) node {$q_1$};
\draw [black] (65.7,-12.9) circle (3);
\draw (65.7,-12.9) node {$q_5$};
\draw [black] (65.7,-12.9) circle (2.4);
\draw [black] (63.8,-33.4) -- (59,-33.4);
\fill [black] (59,-33.4) -- (59.8,-33.9) -- (59.8,-32.9);
\draw (61.4,-32.9) node [above] {$a$};
\draw [black] (53,-33.4) -- (47.4,-33.4);
\fill [black] (47.4,-33.4) -- (48.2,-33.9) -- (48.2,-32.9);
\draw (50.2,-32.9) node [above] {$a$};
\draw [black] (41.4,-33.4) -- (34.9,-33.4);
\fill [black] (34.9,-33.4) -- (35.7,-33.9) -- (35.7,-32.9);
\draw (38.15,-32.9) node [above] {$a$};
\draw [black] (46.2,-12.9) -- (51.5,-12.9);
\fill [black] (51.5,-12.9) -- (50.7,-12.4) -- (50.7,-13.4);
\draw (48.85,-13.4) node [below] {$a$};
\draw [black] (35.7,-12.9) -- (40.2,-12.9);
\fill [black] (40.2,-12.9) -- (39.4,-12.4) -- (39.4,-13.4);
\draw (37.95,-13.4) node [below] {$a$};
\draw [black] (3.2,-12.9) -- (7.1,-12.9);
\fill [black] (7.1,-12.9) -- (6.3,-12.4) -- (6.3,-13.4);
\draw [black] (13.1,-12.9) -- (18,-12.9);
\fill [black] (18,-12.9) -- (17.2,-12.4) -- (17.2,-13.4);
\draw (15.55,-13.4) node [below] {$a$};
\draw [black] (24,-12.9) -- (29.7,-12.9);
\fill [black] (29.7,-12.9) -- (28.9,-12.4) -- (28.9,-13.4);
\draw (26.85,-13.4) node [below] {$a$};
\draw [black] (57.5,-12.9) -- (62.7,-12.9);
\fill [black] (62.7,-12.9) -- (61.9,-12.4) -- (61.9,-13.4);
\draw (60.1,-13.4) node [below] {$a$};
\draw [black] (66.409,-15.815) arc (11.87518:-5.73225:47.804);
\fill [black] (67.19,-30.43) -- (67.77,-29.68) -- (66.78,-29.58);
\draw (67.94,-23.06) node [right] {$a$};
\draw [black] (34.47,-31.84) -- (63.13,-14.46);
\fill [black] (63.13,-14.46) -- (62.19,-14.44) -- (62.71,-15.3);
\draw (49.74,-23.65) node [below] {$a$};
\end{tikzpicture}
\end{center}

Так же построим автомат для языка $B = \{ a^{503n + 29}, n \geq 0 \}$

\begin{center}
\begin{tikzpicture}[scale=0.2]
\tikzstyle{every node}+=[inner sep=0pt]
\draw [black] (10.1,-12.9) circle (3);
\draw (10.1,-12.9) node {$q_0$};
\draw [black] (32.7,-12.9) circle (3);
\draw (32.7,-12.9) node {$q_2$};
%\draw [black] (43.2,-12.9) circle (3);
\draw (43.2,-12.9) node {$\dots$};
\draw [black] (54.5,-12.9) circle (3);
\draw (54.5,-12.9) node {$q_{28}$};
\draw [black] (66.8,-33.4) circle (3);
\draw (66.8,-33.4) node {$Q_1$};
\draw [black] (56,-33.4) circle (3);
\draw (56,-33.4) node {$Q_2$};
%\draw [black] (44.4,-33.4) circle (3);
\draw (44.4,-33.4) node {$\dots$};
\draw [black] (31.9,-33.4) circle (3);
\draw (31.9,-33.4) node {$Q_{502}$};
\draw [black] (21,-12.9) circle (3);
\draw (21,-12.9) node {$q_1$};
\draw [black] (65.7,-12.9) circle (3);
\draw (65.7,-12.9) node {$q_{29}$};
\draw [black] (65.7,-12.9) circle (2.4);
\draw [black] (63.8,-33.4) -- (59,-33.4);
\fill [black] (59,-33.4) -- (59.8,-33.9) -- (59.8,-32.9);
\draw (61.4,-32.9) node [above] {$a$};
\draw [black] (53,-33.4) -- (47.4,-33.4);
\fill [black] (47.4,-33.4) -- (48.2,-33.9) -- (48.2,-32.9);
\draw (50.2,-32.9) node [above] {$a$};
\draw [black] (41.4,-33.4) -- (34.9,-33.4);
\fill [black] (34.9,-33.4) -- (35.7,-33.9) -- (35.7,-32.9);
\draw (38.15,-32.9) node [above] {$a$};
\draw [black] (46.2,-12.9) -- (51.5,-12.9);
\fill [black] (51.5,-12.9) -- (50.7,-12.4) -- (50.7,-13.4);
\draw (48.85,-13.4) node [below] {$a$};
\draw [black] (35.7,-12.9) -- (40.2,-12.9);
\fill [black] (40.2,-12.9) -- (39.4,-12.4) -- (39.4,-13.4);
\draw (37.95,-13.4) node [below] {$a$};
\draw [black] (3.2,-12.9) -- (7.1,-12.9);
\fill [black] (7.1,-12.9) -- (6.3,-12.4) -- (6.3,-13.4);
\draw [black] (13.1,-12.9) -- (18,-12.9);
\fill [black] (18,-12.9) -- (17.2,-12.4) -- (17.2,-13.4);
\draw (15.55,-13.4) node [below] {$a$};
\draw [black] (24,-12.9) -- (29.7,-12.9);
\fill [black] (29.7,-12.9) -- (28.9,-12.4) -- (28.9,-13.4);
\draw (26.85,-13.4) node [below] {$a$};
\draw [black] (57.5,-12.9) -- (62.7,-12.9);
\fill [black] (62.7,-12.9) -- (61.9,-12.4) -- (61.9,-13.4);
\draw (60.1,-13.4) node [below] {$a$};
\draw [black] (66.409,-15.815) arc (11.87518:-5.73225:47.804);
\fill [black] (67.19,-30.43) -- (67.77,-29.68) -- (66.78,-29.58);
\draw (67.94,-23.06) node [right] {$a$};
\draw [black] (34.47,-31.84) -- (63.13,-14.46);
\fill [black] (63.13,-14.46) -- (62.19,-14.44) -- (62.71,-15.3);
\draw (49.74,-23.65) node [below] {$a$};
\end{tikzpicture}
\end{center}

Теперь сделаем автоматы всюду определёнными добавляя отсутствующие переходы в дополнительное непринимающее состояние $q_k$. В том числе из $q_k$ в $q_k$.\\
Теперь построим автомат $\C$ для языка $L = A \cap B$. Используя следующую конструкцию:
\begin{itemize}
	\item $Q_\C = Q_\A \times Q_\B$;
	\item $q_0^{\C} = (q_0^{\A},q_0^{\B})$;
	\item $\forall \sigma \in \Sigma : \delta_\C((q_\A,q_\B), \sigma) = (\delta_\A(q_\A, \sigma), \delta_\B(q_\B, \sigma ) )$;
	\item  $F_\C = F_\A\times Q_\B \cup Q_\A \times F_\B $.
\end{itemize} 
Так как существует ДКА распознающий $L$, то $L$ является регулярным языком.\\
 
2. Этот язык регулярный, так как в регулярном языке разность длин двух последовательных слов из регулярного языка ограниченна линейной функцией, что следует из леммы о накачке. Рассмотрим эту разность: $200(n+1)^2 + 1 - 200n^2 -1 = 400n + 200.$\\

\section*{Задача 4.}

\begin{tabular}{cccc}
Макросост. & сост НКА & 0 & 1 \\
$Q_0$ & $q_0$ & $Q_1$ & $Q_0$ \\
$Q_1$ & $q_0, q_1$ & $Q_1$ & $Q_2$ \\
$Q_2$ & $q_0, q_2$ & $Q_3$ & $Q_0$ \\
$Q_3$ & $q_0, q_1, q_3$ & $Q_4$ & $Q_2$ \\
$Q_4$ & $q_0, q_1, q_4$ & $Q_4$ & $Q_5$ \\
$Q_5$ & $q_0, q_2, q_4$ & $Q_6$ & $Q_7$ \\
$Q_6$ & $q_0, q_1, q_3, q_4$ & $Q_4$ & $Q_5$ \\
$Q_7$ & $q_0, q_4$ & $Q_4$ & $Q_7$ \\
\end{tabular}

\begin{center}
\begin{tikzpicture}[scale=0.2]
\tikzstyle{every node}+=[inner sep=0pt]
\draw [black] (9.4,-9) circle (3);
\draw (9.4,-9) node {$q_0$};
\draw [black] (22,-9) circle (3);
\draw (22,-9) node {$q_1$};
\draw [black] (35.1,-9) circle (3);
\draw (35.1,-9) node {$q_2$};
\draw [black] (56.7,-9) circle (3);
\draw (56.7,-9) node {$q_4$};
\draw [black] (56.7,-9) circle (2.4);
\draw [black] (67.7,-24.2) circle (3);
\draw (67.7,-24.2) node {$q_7$};
\draw [black] (67.7,-24.2) circle (2.4);
\draw [black] (56.7,-24.2) circle (3);
\draw (56.7,-24.2) node {$q_5$};
\draw [black] (56.7,-24.2) circle (2.4);
\draw [black] (45.8,-9) circle (3);
\draw (45.8,-9) node {$q_3$};
\draw [black] (45.8,-24.2) circle (3);
\draw (45.8,-24.2) node {$q_6$};
\draw [black] (45.8,-24.2) circle (2.4);
\draw [black] (12.4,-9) -- (19,-9);
\fill [black] (19,-9) -- (18.2,-8.5) -- (18.2,-9.5);
\draw (15.7,-9.5) node [below] {$a$};
\draw [black] (25,-9) -- (32.1,-9);
\fill [black] (32.1,-9) -- (31.3,-8.5) -- (31.3,-9.5);
\draw (28.55,-9.5) node [below] {$b$};
\draw [black] (8.077,-6.32) arc (234:-54:2.25);
\draw (9.4,-1.75) node [above] {$b$};
\fill [black] (10.72,-6.32) -- (11.6,-5.97) -- (10.79,-5.38);
\draw [black] (3.4,-9) -- (6.4,-9);
\fill [black] (6.4,-9) -- (5.6,-8.5) -- (5.6,-9.5);
\draw [black] (20.677,-6.32) arc (234:-54:2.25);
\draw (22,-1.75) node [above] {$a$};
\fill [black] (23.32,-6.32) -- (24.2,-5.97) -- (23.39,-5.38);
\draw [black] (38.1,-9) -- (42.8,-9);
\fill [black] (42.8,-9) -- (42,-8.5) -- (42,-9.5);
\draw (40.45,-8.5) node [above] {$a$};
\draw [black] (33.193,-11.31) arc (-45.08943:-134.91057:15.5);
\fill [black] (11.31,-11.31) -- (11.52,-12.23) -- (12.23,-11.52);
\draw (22.25,-16.33) node [below] {$b$};
\draw [black] (48.8,-9) -- (53.7,-9);
\fill [black] (53.7,-9) -- (52.9,-8.5) -- (52.9,-9.5);
\draw (51.25,-9.5) node [below] {$a$};
\draw [black] (37.148,-6.848) arc (122.45442:57.54558:6.154);
\fill [black] (37.15,-6.85) -- (38.09,-6.84) -- (37.55,-6);
\draw (40.45,-5.39) node [above] {$b$};
\draw [black] (55.377,-6.32) arc (234:-54:2.25);
\draw (56.7,-1.75) node [above] {$a$};
\fill [black] (58.02,-6.32) -- (58.9,-5.97) -- (58.09,-5.38);
\draw [black] (56.7,-12) -- (56.7,-21.2);
\fill [black] (56.7,-21.2) -- (57.2,-20.4) -- (56.2,-20.4);
\draw (56.2,-16.6) node [left] {$b$};
\draw [black] (53.7,-24.2) -- (48.8,-24.2);
\fill [black] (48.8,-24.2) -- (49.6,-24.7) -- (49.6,-23.7);
\draw (51.25,-23.7) node [above] {$a$};
\draw [black] (59.7,-24.2) -- (64.7,-24.2);
\fill [black] (64.7,-24.2) -- (63.9,-23.7) -- (63.9,-24.7);
\draw (62.2,-24.7) node [below] {$b$};
\draw [black] (47.55,-21.76) -- (54.95,-11.44);
\fill [black] (54.95,-11.44) -- (54.08,-11.8) -- (54.89,-12.38);
\draw (51.84,-17.98) node [right] {$a$};
\draw [black] (54.38,-26.066) arc (-63.45237:-116.54763:7.004);
\fill [black] (54.38,-26.07) -- (53.44,-25.98) -- (53.89,-26.87);
\draw (51.25,-27.3) node [below] {$b$};
\draw [black] (70.688,-24.216) arc (117.43495:-170.56505:2.25);
\draw (74.43,-28.33) node [right] {$b$};
\fill [black] (69.51,-26.58) -- (69.43,-27.52) -- (70.32,-27.06);
\draw [black] (65.94,-21.77) -- (58.46,-11.43);
\fill [black] (58.46,-11.43) -- (58.52,-12.37) -- (59.33,-11.79);
\draw (62.79,-15.22) node [right] {$a$};
\end{tikzpicture}
\end{center}

\section*{Задача 6.}
Определим КМП-автомат для слова $abaa$:
	\begin{itemize}
		\item $Q = \{\; \eps,\; a,\; ab, aba,\; abaa\; \}$;
		\item $q_0 = \eps$;
		\item $ \delta:$\\
		\hspace*{0.5cm} $ \delta(\eps, a) = a$ \\
		\hspace*{0.5cm} $ \delta(\eps, b) = \eps$ \\
		\hspace*{0.5cm} $ \delta(a, a) = a$ \\
		\hspace*{0.5cm} $ \delta(a, b) = ab$ \\
		\hspace*{0.5cm} $ \delta(ab, a) = aba$ \\
		\hspace*{0.5cm} $ \delta(ab, b) = \eps$ \\
		\hspace*{0.5cm} $ \delta(aba, a) = abaa$ \\
		\hspace*{0.5cm} $ \delta(aba, b) = ab$ \\
		\hspace*{0.5cm} $ \delta(abaa, a) = abaa$ \\
		\hspace*{0.5cm} $ \delta(abaa, b) = abaa$ \\  
		\item $F = \{ abaa \}$.
	\end{itemize}
Построим этот автомат.
\begin{center}
\begin{tikzpicture}[scale=0.2]
\tikzstyle{every node}+=[inner sep=0pt]
\draw [black] (8.7,-12.4) circle (3);
\draw (8.7,-12.4) node {$\eps$};
\draw [black] (20,-12.4) circle (3);
\draw (20,-12.4) node {$a$};
\draw [black] (32,-12.4) circle (3);
\draw (32,-12.4) node {$ab$};
\draw [black] (45.2,-12.4) circle (3);
\draw (45.2,-12.4) node {$aba$};
\draw [black] (56.2,-12.4) circle (3);
\draw (56.2,-12.4) node {$abaa$};
\draw [black] (56.2,-12.4) circle (2.4);
\draw [black] (1.9,-12.4) -- (5.7,-12.4);
\fill [black] (5.7,-12.4) -- (4.9,-11.9) -- (4.9,-12.9);
\draw [black] (11.7,-12.4) -- (17,-12.4);
\fill [black] (17,-12.4) -- (16.2,-11.9) -- (16.2,-12.9);
\draw (14.35,-12.9) node [below] {$a$};
\draw [black] (23,-12.4) -- (29,-12.4);
\fill [black] (29,-12.4) -- (28.2,-11.9) -- (28.2,-12.9);
\draw (26,-12.9) node [below] {$b$};
\draw [black] (35,-12.4) -- (42.2,-12.4);
\fill [black] (42.2,-12.4) -- (41.4,-11.9) -- (41.4,-12.9);
\draw (38.6,-12.9) node [below] {$a$};
\draw [black] (48.2,-12.4) -- (53.2,-12.4);
\fill [black] (53.2,-12.4) -- (52.4,-11.9) -- (52.4,-12.9);
\draw (50.7,-12.9) node [below] {$a$};
\draw [black] (54.877,-9.72) arc (234:-54:2.25);
\draw (56.2,-5.15) node [above] {$a,b$};
\fill [black] (57.52,-9.72) -- (58.4,-9.37) -- (57.59,-8.78);
\draw [black] (18.677,-9.72) arc (234:-54:2.25);
\draw (20,-5.15) node [above] {$a$};
\fill [black] (21.32,-9.72) -- (22.2,-9.37) -- (21.39,-8.78);
\draw [black] (30.231,-14.815) arc (-42.61929:-137.38071:13.428);
\fill [black] (10.47,-14.82) -- (10.64,-15.74) -- (11.38,-15.07);
\draw (20.35,-19.65) node [below] {$b$};
\draw [black] (33.597,-9.887) arc (135.34474:44.65526:7.033);
\fill [black] (33.6,-9.89) -- (34.51,-9.67) -- (33.8,-8.97);
\draw (38.6,-7.3) node [above] {$b$};
\draw [black] (7.7,-9.584) arc (227.29016:-60.70984:2.25);
\draw (9.74,-5.11) node [above] {$b$};
\fill [black] (10.33,-9.89) -- (11.24,-9.64) -- (10.5,-8.97);
\end{tikzpicture}
\end{center}

\section*{Задача 8.}
Алгоритм будет работать со строкой $\o=abba\#abbbababbab$. На каждом шаге он будет заполнять одну клетку таблицы длины 16. В $i$ \textit{ую} клетку таблицы заносится значение длины префикс-функции от слова  $\o[0, i]$. Заполним такую таблицу:\\

\begin{tabular}{|c|c|c|c|c|c|c|c|c|c|c|c|c|c|c|c|}
\hline
0 & 0 & 0 & 1 & 0 & 1 & 2 & 3 & 0 & 1 & 2 & 1 & 2 & 3 & 4 & 0\\
\hline
\end{tabular}\\

Длина искомого подслова 4, в получившейся таблице есть 4ка, значит, подслово входит в слово.

\end{document}
